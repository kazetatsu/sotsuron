\chapter{序論}

\section{研究背景}
スマートタグとは貴重品などに取り付け, 紛失した際に捜索を補助するデバイスである. 
補助の方法はタグが音を鳴らす方法と付近のスマートフォンと無線通信する方法が一般的である. 
近年こうした製品が普及しており, なくしものの対策に需要があることを示唆している. 
しかし特性上タグの取り付けが困難な物もある. 
例えば入れ歯は口内で使用するために衛生や防水, 日常生活の邪魔になるなどの理由で取り付けが難しい. 

\section{関連研究}
タグを必要としないなくしもの対策として, 草野\index{くさの@草野}らが提案する家庭内での移動マニピュレータによるベイズ推定と自然言語処理を用いた物品捜索手法\cite{kusano}がある. 
この手法ではまず捜索範囲をいくつかのエリアに区切る. 
そして各エリアごとに過去そのエリアを捜索し捜索対象を発見した回数$\alpha$, 発見できなかった回数$\beta$を用い, 
捜索対象の存在確率を$ \alpha / (\alpha + \beta)$と定義し, 存在確率が最大のエリアを捜索する. 
また存在確率が最大のエリアが複数ある場合は移動マニピュレータが位置するエリアからの距離, ならびに自然言語処理によるエリア内の家具の名前と捜索対象の名前の類似度も考慮する. 

\begin{figure}[H]
    \begin{center}
        \includegraphics[width=0.8\linewidth]{figs/kusano.jpg}
        \caption{Existence probability of target object}
        \label{fig:kusano}
    \end{center}
\end{figure}

この手法は捜索する主体が移動マニピュレータであることと密接に結びついている. 
しかし現在, そうしたロボットはスマートタグの代替としては高価であり一般的な家庭での利用は難しい. 

また, 人がなくしものを捜索する際に捜索対象の所有者(以下, 単に所有者という)がよく使う場所を重点的に捜索する等, 所有者の行動を手がかりにする. 
一方この手法の場合, なくしものは発生頻度が高くないため$\alpha$が同じエリアが複数ある状況が長く続くことが想定される. 
こうした状況ではエリア間の距離と単語の類似度を重視することになるが, これらは所有者の行動を反映していない. 

\section{研究目的}
利用者のなくしものが発生していない状況下での行動を反映することで最適化がより進む
確率論でモデル化
なくしものの位置の確率分布を計算
