\chapter{提案手法}\label{chap:method}

\section{概要}
経験則をもとになくしものの発生原理をモデル化
手に持っていたものを置いて次の動作に遷移
置いたことを忘れる
例
郵便受けから新聞紙をとろうと手に持っていたスマートフォンを床に置いたらそのまま忘れた

モデル化
人は動作をノードとするマルコフ連鎖に従って行動
異なる動作に遷移する際, 確率的になくしもの発生
遷移を繰り返せばいつかはなくしものが発生
どの動作に遷移する際になくしものが発生しやすい?
各動作でなくしものが発生する確率

動作ごとに利用者がいる位置の傾向を表す分布を用意
なくしもの発生確率を重みとして利用者位置分布を加算
和をなくしものの位置の分布とする

利用者位置分布とマルコフ連鎖を通じてなくしものが発生していない状況下での利用者の行動を反映

\section{定式化}
捜索範囲を$ X \subseteq \mathbb{R}^d \  (d=2または3)$とおく.
マルコフ連鎖の状態数すなわち動作の数を$ n $,各動作を$ M_i (i=1,2,\cdots ,n)$,$ M_i $から$ M_j $への遷移確率を$ a_{i j} $とおく.
またマルコフ連鎖はエルゴード的であるとする.\cite{funaki}
% \footnote{エルゴード的であるとはマルコフ連鎖を有向グラフとみなしたとき,任意の$ i $,$ j $に対し$ S_i $から$ S_j $への経路が存在し,かつ任意の$ i $に対し$ S_i $から$ S_i $への閉路の長さの最大公約数が1であることである}
初期動作が$ M_i $である確率を$ s_i $とおく.
$ t(t=0,1,2,\cdots) $回目の遷移の後の動作を$ m_t $とおく.ただし$ m_0 $は初期動作とする.
なくしものが何回目の遷移で発生したか,すなわち$ t - 1 $回目の遷移までなくしものが発生せず$ t $回目でなくしものが発生するような$ t $を$ \tau $とおく.
$ M_i $から$ M_j $へ遷移する際になくしものが発生する確率,すなわち$ \mathrm{P}(\tau = t | \tau \ge t , m_t = M_j , m_{t - 1} = M_i) $を$ M_i $から$ M_j $への遷移失敗確率とよぶ.
この確率は$ i = j $のとき$ 0 $,$ i \ne j $のとき$ M_j $にのみ依存した$ 0 $より真に大きい値をとると仮定しその値を$ \theta_j $とおく.
$ M_i $に対応する利用者の位置分布を$ h_i:X \rightarrow [0,\infty] $とし$ \int_X h_i = 1 $を満たすとする.積分条件は加重和をとる際に余分な重みが加わらないようにするためである.

以上を用いて$ M_i $においてなくしものが発生する確率$ p_i = \mathrm{P}(\tau < \infty , m_{\tau} = M_i) $は
\begin{equation} \label{eq:p}
p = L(I-K)^{-1}s
\end{equation}
となる.ただし
\begin{align*}
\arraycolsep1pt
p &= (p_1,p_2,\cdots,p_n)^\mathrm{T} \\
L &=
\begin{pmatrix}
    0 & \theta_1 a_{2 1} & \cdots & \theta_1 a_{n 1} \\
    \theta_2 a_{1 2} & 0 & \cdots & \theta_2 a_{n 2} \\
    \vdots & \vdots & \ddots & \vdots \\
    \theta_n a_{1 n} & \theta_n a_{2 n} & \cdots & 0 \\
\end{pmatrix}
\\
I &= n行n列の単位行列 \\
K &=
\begin{pmatrix}
    a_{1 1} & (1 - \theta_1) a_{2 1} & \cdots & (1 - \theta_1) a_{n 1} \\
    (1 - \theta_2) a_{1 2} & a_{2 2} & \cdots & (1 - \theta_2) a_{n 2} \\
    \vdots & \vdots & \ddots & \vdots \\
    (1 - \theta_n) a_{1 n} & (1 - \theta_n) a_{2 n} & \cdots & a_{n n} \\
\end{pmatrix}
\\
s &= (s_1,s_2,\cdots,s_n)^\mathrm{T}
\end{align*}
とおいた.


ここに書いてある方法を使えば, 秒速で秒速で1億円稼ぐ男になれます. なれません. 


\section{手法の概要}

図に書くと図\ref{fig:vq_map_128part}っていう感じ. 
式で書くとだいたい以下のような感じになるんじゃないんかなー. 
式(\ref{eq:j})が肝. 

\begin{figure}[h]
        \begin{center}
        \includegraphics[width=1.0\linewidth]{figs/vq_map_128part.eps}
        \caption{Representative Vectors of the $N_c = 128$ Map}
        \label{fig:vq_map_128part}
        \end{center}
\end{figure}




\begin{align}
s_0, a(t_0), s(t_1), a(t_1), s(t_2), a(t_2), \dots, a(t_{T-1}), s_\text{f} \quad (s_0 = s(t_0), s_\text{f} = s(t_T)). 
\end{align}
\begin{align}
& s_0, \pi(s_0), s(t_1), \pi(s(t_1)), s(t_2), \pi(s(t_2)), \dots, \pi(s(t_{T-1})), s_\text{f}
\end{align}
\begin{align}
\pi &: \mathcal{S} \to \mathcal{A} \label{eq:policy_state_action_sequence}
\end{align}
\begin{align}
\mathcal{S} &= \{s_i | i=0,1,2,\dots,N-1 \}, \text{ and} \\
\mathcal{A} &= \{a_j | j=0,1,2,\dots,M-1 \}
\end{align}
\begin{align}
\pi : \mathcal{S} - \mathcal{S}_\text{f} \to \mathcal{A}. \label{eq:policy}
\end{align}
\begin{align}
\dot{\V{x}}(t) &= \V{f}[\V{x}(t),\V{u}(t)], \quad \V{x}(0) = \V{x}_0, \quad t \in [0,t_\text{f}].\label{eq:system} \\
&  \nonumber 
\end{align}
\begin{align}
g[\V{x}(t), \V{u}(t)] \in \Re \quad (t \in [0,t_\text{f}]). \label{eq:evaluation_function}
\end{align}
\begin{align}
J[\V{u}] = \int_{0}^{t_\text{f}} g[\V{x}(t), \V{u}(t)] dt + V(\V{x}_\text{f}).  \label{eq:functional}
\end{align}
\begin{align}
\max_{\V{u}:[0,t_\text{f}) \to \Re^m} J[\V{u};\V{x}_0].  \label{eq:optimal_control_problem}
\end{align}
\begin{align}
\V\pi^*: \Re^n \to \Re^m
\end{align}
\begin{align}
\max_{\V{u}:[0,t_\text{f}) \to \Re^m} J[\V{u};\V{x}_0] &= \max_{\V{u}:[0,t') \to \Re^m} \int_{0}^{t'} g[\V{x}(t), \V{u}(t)] dt \nonumber \\ &+ \max_{\V{u}:[t',t_\text{f}) \to \Re^m} \int_{t'}^{t_\text{f}} g[\V{x}(t), \V{u}(t)] dt + V(\V{x}_\text{f}) \nonumber \\
	&= \max_{\V{u}:[0,t') \to \Re^m} \int_{0}^{t'} g[\V{x}(t), \V{u}(t)] dt + \max_{\V{u}:[t',t_\text{f}) \to \Re^m} J[\V{u};\V{x}(t')]. \label{eq:j}
\end{align}
\begin{align}
V^{\V\pi}(\V{x}) &= J[\V{u};\V{x}], \label{eq:def_of_value} \\
&\text{ where } \V{u}(t) = \V\pi(\V{x}(t)), \ 0\le t \le t_\text{f}. \nonumber 
\end{align}
\begin{align}
\mathcal{P}_{ss'}^a &= P[s(t_{i+1}) = s' | s(t) = s,a(t) = a], \label{eq:state_transition}\\
&(\forall t \in \{t_0,t_1,\dots,t_{T-1}\}, \forall s \in \mathcal{S} - \mathcal{S}_\text{f}, \text{ and } \forall s' \in \mathcal{S}). \nonumber
\end{align}
\begin{align}
\mathcal{R}_{ss'}^a \in \Re
\end{align}
\begin{align}
J[a;s(t_0)] = J[a(0),a(1),\dots,a(t_{T-1})] = \sum_{i=0}^{T-1} \mathcal{R}_{s(t_i)s(t_{i+1})}^{a(t_i)} + V(s(t_T)),
\end{align}
\begin{align}
\max J[a;s(t_0)]. 
\end{align}
\begin{align}
J^{\V{\pi}} = \int_\mathcal{X} p(\V{x}_0) J[\V{u};\V{x}_0] d\V{x}_0 \quad\Big(\V{u}(t) = \V\pi(\V{x}(t))\Big),\label{eq:eval_general}
\end{align}
\begin{align}
\dfrac{\partial V(\V{x})}{\partial t} = \max_{\V{u}\in\mathcal{U}} \left[g[\V{x},\V{u}] + \dfrac{\partial V(\V{x})}{\partial\V{x}} \V{f}[\V{x},\V{u}] \right].\label{eq:hjb}
\end{align}
\begin{align}
U_\text{att}(\V{x}) = \dfrac{1}{2} \xi \rho^2 (\V{x})
\end{align}
\begin{align}
U_\text{rep}(\V{x}) =
\begin{cases}
\dfrac{1}{2}\eta \left( \dfrac{1}{\rho(\V{x})} - \dfrac{1}{\rho_0} \right)^2 &\text{if } \rho(\V{x}) \le \rho_0, \\
0 &\text{if } \rho(\V{x}) > \rho_0,
\end{cases}
\end{align}
\begin{align}
U(\V{x}) = U_\text{att}(\V{x}) + U_\text{rep}(\V{x})
\end{align}
\begin{align}
\V{F}(\V{x}) &= - (\partial U/\partial x_1,\partial U/\partial x_2,\dots,\partial U/\partial x_n)^T \nonumber \\
&= -\nabla U({\V{x}}). 
\end{align}
\begin{align}
V(\V{x};\theta_1,\theta_2,\dots,\theta_{N_\theta}) \nonumber
\end{align}
\begin{align}
\phi_i(\V{x}) &= \exp \left\{-\dfrac{1}{2}(\V{x} - \V{c}_i)^t M_i (\V{x} - \V{c}_i) \right\},
\end{align}
\begin{align}
b_i(\V{x}) &= \dfrac{\phi_i(\V{x})}{\sum_{j=1}^{N_\phi} \phi_j(\V{x})}, \ (N_\phi: \text{ number of RBFs in the space})
\end{align}
\begin{align}
V(\V{x}) &= \sum_{i=1}^{N_\phi} \nu_i b_i(\V{x}). \label{eq:rbf_weighted_sum}
\end{align}
\begin{align}
\phi_i(x) &= \exp \left\{-\dfrac{1}{2}(x - i)^2 \right\} \nonumber
\end{align}

\begin{align}
V(\V{x}) = \sum_{i=0}^3 w_i V(\V{x}_i)
\end{align}


\begin{table}[htbp]
        \begin{center}
	\caption{謎のパラメータ}
        \label{table:parameter_value}
\begin{footnotesize}
\begin{minipage}{12em}
        \begin{tabular}{c|rl}
\multicolumn{3}{c}{(a)}\\
        \thline
parameter & \multicolumn{2}{c}{value} \\
        \hline
$\ell_1,\ell_2$ & 1.0&[m] \\
$\ell_{c1},\ell_{c1}$ & 0.50&[m] \\
$m_1,m_2$ & 1.0&[kg] \\
$I_1,I_2$ & 1.0&[kg m$^2$] \\
$g$ & 9.8&[m/s$^2$] \\
   \thline
  \end{tabular}
\end{minipage}
\hspace{2em}
\begin{minipage}{12em}
        \begin{tabular}{c|l}
\multicolumn{2}{c}{(b)}\\
        \thline
variable & domain \\
        \hline
$\theta_1$ & $(-\infty,\infty)$ \\
$\theta_2$ & $(-\infty,\infty)$ \\
$\dot\theta_1$ & $[-720,720]$[deg/s] \\
$\dot\theta_2$ & $[-1620,1620]$[deg/s] \\
\hline
$\tau$ & $-1,0,$ or $1$[Nm] \\
   \thline
  \end{tabular}
\end{minipage}
\end{footnotesize}
  \end{center}
\end{table}


